\documentclass[12pt,a4paper]{article}
\usepackage[T1]{fontenc}
%\usepackage[utf8]{inputenc}
\usepackage[latin1]{inputenc}

\usepackage[french]{babel}
\frenchbsetup{StandardLists=true}
\usepackage{enumitem}

\usepackage{hyperref}
\usepackage{amsmath}
\usepackage{amsfonts}
\usepackage{amssymb}
\usepackage{graphicx}
\usepackage{layout}
\newenvironment{script}%
{\vspace{-22pt}\begin{quote}\obeylines\noindent\ignorespaces\tt}%
{\par\noindent\ignorespacesafterend\end{quote}}

\newcounter{question}
\newenvironment{question}[1][]%
{\vspace{5pt}\refstepcounter{question}\vspace{2pt}\textbf{Question \thequestion. #1}%
	\begin{quote}\obeylines\ignorespaces}%
	{\par\noindent\ignorespacesafterend\end{quote}}

\newenvironment{remarque}%
{\vspace{2pt}\textit{Remarque}\begin{quote}\obeylines\ignorespaces}%
	{\par\noindent\ignorespacesafterend\end{quote}}

\newenvironment{commandes}%
{\vspace{2pt}\textbf{Commandes}\begin{quote}\obeylines\ignorespaces}%
	{\par\noindent\ignorespacesafterend\end{quote}}

\newenvironment{liste}%
{\begin{itemize}[parsep=1pt]}%
{\end{itemize}}

\parindent=0pt
\oddsidemargin=0pt
\marginparwidth=0pt
\textwidth=460pt
\textheight=650pt
\headsep=0pt


\begin{document}
%    \layout
    
    \begin{center}
    	\LARGE TP 4. Programmation hybride MPI - OpenMP
    \end{center}

	\section*{Pr�paration}
    
    R�cup�rer l'archive compress�e {\tt TP4.tar.gz} et extraire les fichiers qui sont contenus dans cette archive:
    \begin{script}
        cd <repertoire dans votre espace de travail>
        tar xvfz TP4.tar.gz
    \end{script}

    Se placer dans le r�pertoire {\tt TP4}:
    \begin{script}
        cd TP4
    \end{script}

    et pr�parer les compilations dans les points suivants avec les commandes ci-dessous:
    \begin{script}
        mkdir -p build
        cd build
        cmake ../src
        make install
        cd ..
    \end{script}
	
\section{Exemple MPI}

Le fichier {\tt src/sinus\_mpi/sinus.cxx} contient une version distribu�e avec MPI sur plusieurs processus.

\begin{question}
   Comparer les temps d'ex�cution de la version s�quentielle et de cette version en tapant les commandes
    \begin{script}
		./build/sinus\_seq/sinus\_seq 40000
		mpirun -n 3 ./build/sinus\_mpi/sinus\_mpi 40000
	\end{script}
\end{question}

\begin{remarque}
On ne peut pas tirer de conclusions d�finitives sur la comparison MPI / OpenMP sur un exemple aussi petit.
\end{remarque}

\section{Exemple MPI - OpenMP ``grain fin''}

Le fichier {\tt src/sinus\_mpi\_openmp\_fine/sinus.cxx} contient une version distribu�e avec MPI sur plusieurs processus, chacun contenant plusieurs threads avec OpenMP (version grain fin). 

\begin{question}
   Comparer les fichiers {\tt src/sinus\_mpi/sinus.cxx} et {\tt src/sinus\_mpi\_openmp\_fine/sinus.cxx}.

   Ex�cuter le code MPI sur 4 processus avec la commande
    \begin{script}
      mpirun -n 4 ./build/sinus\_mpi/sinus\_mpi 40000
    \end{script}

  et le code MPI\_OpenMP (grain fin) sur 2 processus, chacun avec 2 threads, avec la commande
    \begin{script}
      mpirun -n 2 -x OMP\_NUM\_THREADS=2  \textbackslash
      ./build/sinus\_mpi\_openmp\_fine/sinus\_mpi\_openmp\_fine 40000
    \end{script}
  
\end{question}

\section{Exemple MPI - OpenMP ``grain grossier''}

Le fichier {\tt src/sinus\_mpi\_openmp\_coarse/sinus.cxx} contient une version distribu�e avec MPI sur plusieurs processus, chacun contenant plusieurs threads avec OpenMP (version grain grossier). 

\begin{question}
   Comparer les fichiers {\tt src/sinus\_mpi/sinus.cxx} et {\tt src/sinus\_mpi\_openmp\_coarse/sinus.cxx}.

   Ex�cuter le code MPI sur 4 processus avec la commande
    \begin{script}
      mpirun -n 4 ./build/sinus\_mpi/sinus\_mpi 40000
    \end{script}

  et le code MPI\_OpenMP (grain fin) sur 2 processus, chacun avec 2 threads, avec la commande
    \begin{script}
      mpirun -n 2 -x OMP\_NUM\_THREADS=2  \textbackslash
      ./build/sinus\_mpi\_openmp\_fine/sinus\_mpi\_openmp\_coarse 40000
    \end{script}
  
\end{question}

\section{Versions hybrides MPI-OpenMP du mini-code}

On fournit 3 versions du code : en m�moire distribu�e avec MPI et 2 versions en m�moire partag�e avec OpenMP.

\begin{question}
Construire une version qui combine les 2 types de parall�lisme MPI et OpenMP ``grain fin''.
\end{question}

\begin{question}
Construire une version qui combine les 2 types de parall�lisme MPI et OpenMP ``grain grossier''.
\end{question}

\end{document}
