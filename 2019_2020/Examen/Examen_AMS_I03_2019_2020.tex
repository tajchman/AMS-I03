\documentclass[12pt]{report}
\setlength{\textwidth}{450pt}
\setlength{\textheight}{600pt}
\setlength{\oddsidemargin}{0pt}
\usepackage[T1]{fontenc}
\usepackage{color}
\usepackage{listings}
\usepackage{fancyhdr}
\pagestyle{fancy}
\parindent 0pt
\lstset{ 
   keepspaces=true,
   language=C++,
   numbers=left
}

\lstnewenvironment{algorithm}[1][] %defines the algorithm listing environment
{   
	\lstset{ %this is the stype
		mathescape=true,
		numbers=left, 
		numberstyle=\tiny,
		basicstyle=\normalsize, 
		keywordstyle=\color{black}\bfseries\em,
		keywords={,input, output, return, datatype, function, in, if, else, for, to, while, begin, end, } %add the keywords you want, or load a language as Rubens explains in his comment above.
		numbers=left,
		xleftmargin=.04\textwidth,
		#1 % this is to add specific settings to an usage of this environment (for instnce, the caption and referable label)
	}
}
{}

\def\Frac#1#2{\frac{\displaystyle #1}{\displaystyle #2}}
\newcounter{cptPoints}

\newcounter{cptQuestions}
\newcommand\question[2]{\bigskip\par\addtocounter{cptQuestions}{1}\addtocounter{cptPoints}{#2}{\bf Question #1 n\textsuperscript{o} \thecptQuestions} (#2 \ifnum #2>1 points\else point\fi)\par}

\newcounter{cptProblems}
\newcommand\problem[1]{\bigskip\rule{3cm}{.1pt}\par\addtocounter{cptProblems}{1}{\bf Probl\` eme n\textsuperscript{o} \thecptProblems \ (#1)}\medskip\par}

\begin{document}
	\lhead{\bf ENSTA - Master AMS M2 - Cours I03}
	\rhead{\bf 2019-2020}
	\begin{center}\Large\bf
			Examen du cours AMS-I03\\
			Programmation hybride et multi-c\oe urs\\[0.4cm]
			Vendredi 14 f\'evrier 2020 - dur\'ee 3 heures\\
			Supports de cours autoris\'es.
		\end{center}
	\bigskip
		
	{\bf La syntaxe des lignes de code que vous écrirez ne sera pas évaluée (oubli de ``;'' ou ordre des arguments dans l'appel des fonctions par exemple), par contre, bien insérer les pragmas et appeler les fonctions nécessaires.
	
    Ajouter des lignes commentaires dans le code que vous écrivez.}
	
	\question{de cours}1
	\medskip
	
	 D\'efinir les notions de localit\'e spatiale et localit\'e temporelle.  Donnez un exemple simple pour illustrer chacune de ces deux notions.

\bigskip

\hrule
\medskip

{\bf Partie 1 : parallélisation en mémoire partagée}
\medskip

\question{de cours}1
\medskip

Rappeler les diff\'erences principales entre la programmation OpenMP \guillemotleft grain fin\guillemotright \ (fine-grain) et \guillemotleft gros grain\guillemotright \ (coarse-grain).

	\question{}5
    \medskip

Soient $A$ une matrice triangulaire inférieure ($A_{i,j} = 0$ si $j > i$) de taille $n \times n$, $V$ et $W$ deux vecteurs de taille $n$.

Le pseudo-code suivant calcule le produit matrice vecteur : $w = A v$, en tenant compte de la structure triangulaire inférieure de $A$:

\begin{quote}
\begin{algorithm}
input: matrix A, vector V, int n
output: vector W

for i = 0 to n-1
   s $\gets$ 0.0
   for j = 0 to i
      s $\gets$ s + A(i,j)*V(j)
   end
   W(i) $\gets$ s
end       
\end{algorithm}
\end{quote}

\bigskip
	
\begin{itemize}
		\item Parall\'eliser suivant le mod\`ele ``OpenMP grain fin'' ce code
		\item Expliquer pourquoi le parall\'elisme grain fin ne sera probablement pas optimal.  
		\item Parall\'eliser suivant le mod\`ele ``OpenMP grain grossier'' (ou gros grain) ce code en tenant compte de la structure de $A$
	\end{itemize}


\bigskip

\hrule
\medskip

{\bf Partie 2 : parallélisation hybride}
\medskip

\question{de cours}1
\medskip

Dans le mod\`ele de programmation hybride MPI-OpenMP, d\'ecrivez des avantages esp\'er\'es par rapport \`a une programmation purement MPI et une programmation purement OpenMP. 

\question{}5
\medskip

Le but de cet exercice est de calculer une valeur approch\'ee de $\pi$ par int\'egration num\'erique en mod\`ele de programmation mixte MPI+OpenMP.
\medskip

La machine de calcul dispose de $N$ n\oe uds (machine à 1 processeur), chaque processeurs \'etant compos\'e de $M$ c\oe urs. Le nombre total de c\oe urs est donc de $N\times M$.
\medskip

Soit la formule suivante utilis\'ee pour calculer la valeur du nombre $\pi$ :

$$
\pi = \int^1_0 \Frac{4}{1+x^2}\ dx
$$

Le programme s\'equentiel suivant permet de calculer une valeur approch\'ee de cette int\'egrale en utilisant la m\'ethode du trap\`eze. Cette m\'ethode simple consiste \`a remplir la surface sous la courbe par une s\'erie de petits rectangles. Lorsque la largeur des rectangles tend vers z\'ero, la somme de leur surface tend vers la valeur de l'int\'egrale (et donc vers $\pi$). 

\begin{quote}
\begin{lstlisting}
#include <iostream>

int main() {
  int lNumSteps = 100000000;
  double lStep = 1.0/lNumSteps;
  double lSum = 0.0, x;

  for (int i=0; i<lNumSteps ; ++i ) {
    x = (i+0.5) * lStep;
   lSum += 4.0/(1.0 + x*x);
  }
  double pi = lSum*lStep;

  std::cout.precision(15) ;
  std::cout << "pi =" << pi << std::endl;

  return 0;
}
\end{lstlisting}
\end{quote}

\begin{itemize}
	\item Ajouter une parall\'elisation OpenMP (type ``grain fin'').
	\item Ajouter une parall\'elisation MPI pour obtenir une parall\'elisation hybride.
\end{itemize}

\bigskip

\hrule
\medskip

{\bf Partie 3 : parallélisation sur GPU}
\medskip

\question{de cours}2

\begin{itemize}
	\item D\'ecrivez en quelques lignes, le mod\`ele de parall\'elisme utilis\'e dans un GPU.
	\item Quelles sont les principales diff\'erences entre CUDA et OpenCL (mode de d\'efinition d'un noyau de calcul et ex\'ecution de ce noyau sur un GPU) ?
\end{itemize}

\question{}5

\medskip 
 
Soit $A$ une matrice ``proche'' de la matrice identit\'e $Id$ ($Id_{i,j} = 1$ si $i = j$ et $0$ sinon). L'algorithme suivant calcule une approximation de la matrice inverse de $A$ :

$$
A^{-1} \approx Id + \lim_{k=1}^{K} \ (Id - A)^k
$$

On suppose que cette approximation converge quand $K$ tend vers $\infty$.

\begin{itemize}
	\item \'Ecrire un programme principal en C ou C++, qui impl\'emente cet algorithme. Ce programme ex\'ecutera une s\'erie de noyaux Cuda en s'arr\^etant d\`es que le dernier terme ajout\'e \`a la somme ci-dessus est inf\'erieur \`a une quantit\'e donn\'ee.
	\item On fera attention \`a limiter le nombre de transferts entre la m\'emoire du CPU et celle du GPU.
\end{itemize}
         
\bigskip \rule{3cm}{.1pt}

Total : \thecptPoints \ points
 
\end{document}
