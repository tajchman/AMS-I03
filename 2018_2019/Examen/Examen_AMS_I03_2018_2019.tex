\documentclass[12pt]{report}
\setlength{\textwidth}{450pt}
\setlength{\textheight}{600pt}
\setlength{\oddsidemargin}{0pt}
\usepackage[T1]{fontenc}
\usepackage{listings}
\usepackage{fancyhdr}
\pagestyle{fancy}
\parindent 0pt
\lstset{ 
   keepspaces=true,
   language=C++,
   numbers=left
}

\def\Frac#1#2{\frac{\displaystyle #1}{\displaystyle #2}}
\newcounter{cptPoints}

\newcounter{cptQuestions}
\newcommand\question[2]{\bigskip\par\addtocounter{cptQuestions}{1}\addtocounter{cptPoints}{#2}{\bf Question #1 n\textsuperscript{o} \thecptQuestions} (#2 points)\par}

\newcounter{cptProblems}
\newcommand\problem[1]{\bigskip\rule{3cm}{.1pt}\par\addtocounter{cptProblems}{1}{\bf Probl\` eme n\textsuperscript{o} \thecptProblems \ (#1)}\medskip\par}

\begin{document}
	\lhead{\bf ENSTA - Master AMS M2 - Cours I03}
	\rhead{\bf 2018-2019}
	\begin{center}\Large\bf
			Examen du cours I3\\
			Programmation hybride et multi-c\oe urs\\[0.4cm]
			Vendredi 15 f\'evrier 2019 - dur\'ee 3 heures\\
			Supports de cours autoris\'es.
		\end{center}
	\bigskip
	Dans les questions o\`u on demande d'\'ecrire des lignes de code, les erreurs de syntaxe ne seront pas prises en compte (ponctuation, nom exact des fonctions, ordre des arguments, etc.). Du moment que vous indiquez clairement ce que fait chaque ligne de code ajout\'ee pour r\'epondre aux questions.
		
	\bigskip
	
	\question{}2
	On suppose utiliser la machine qui contient 1 processeur avec 2 c{\oe}urs et chaque c{\oe}ur possède une mémoire cache séparée:
	
	
	\question{}5
	\medskip
	On veut paralléliser avec OpenMP, la fonction C++ :
	\lstinputlisting[language=C++]{Q2/maxlocal.hxx}
	
	\begin{quotation}\noindent%
		Le but de cette fonction est de calculer, sur les composantes d'un vecteur {\tt v}, les indices des {\tt maxima locaux} ($i$ tels que $v_i > v_{i-1}$ et $v_i > v_{i+1}$).
		
		\noindent%
		On suppose que le nombre de maxima locaux est inférieur à la taille du vecteur d'indices {\tt imax}.
	\end{quotation}
	 
 La parallélisation de cette fonction en ajoutant une pragma simple suivant le modèle ``OpenMP grain fin'' est impossible (le compilateur refuse de compiler quand on ajoute la pragma), ou en tout cas compliquée.
 
	 \begin{itemize}
	 	\item Expliquer pourquoi.
	 	\item Modifier la fonction en la parallélisant suivant le modèle ``OpenMP grain grossier''.
	 	\item La version parallèle fournit un résultat correct mais peut-être différent de celui de la version séquentielle. Quelle est cette différence ?
	 	\item Question optionnelle : Indiquer une méthode pour obtenir si possible un résultat identique.
	 \end{itemize}
	 
         
\bigskip \rule{3cm}{.1pt}

Total : \thecptPoints \ points
 
\end{document}
