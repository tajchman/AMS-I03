\documentclass[12pt,a4paper]{article}
\usepackage[T1]{fontenc}
%\usepackage[utf8]{inputenc}
\usepackage[latin1]{inputenc}

\usepackage[french]{babel}
\frenchbsetup{StandardLists=true}
\usepackage{enumitem}

\usepackage{hyperref}
\usepackage{amsmath}
\usepackage{amsfonts}
\usepackage{amssymb}
\usepackage{graphicx}
\usepackage{layout}
\newenvironment{script}%
{\vspace{-15pt}\begin{quotation}\obeylines\noindent\ignorespaces\tt}%
{\par\noindent\ignorespacesafterend\end{quotation}}

\newcounter{question}
\newenvironment{question}[1][]%
{\refstepcounter{question}\vspace{2pt}\textbf{Question \thequestion. #1}%
	\begin{quote}\obeylines\ignorespaces}%
	{\par\noindent\ignorespacesafterend\end{quote}}

\newenvironment{remarque}%
{\vspace{2pt}{\it Remarque}\begin{quote}\obeylines\ignorespaces}%
	{\par\noindent\ignorespacesafterend\end{quote}}


\newenvironment{liste}[1]%
{\begin{itemize}[label=#1, parsep=50pt]}%
{\end{itemize}}

\parindent=0pt
\oddsidemargin=0pt
\marginparwidth=0pt
\textwidth=460pt
\textheight=650pt
\headsep=0pt


\begin{document}
%    \layout
    
    \begin{center}
    	\LARGE TP 8. Outils de plus haut niveau pour la programmation parall�le
    \end{center}

	\section*{Pr�paration}
    Se connecter sur la machine \texttt{rhum} � l'ENSTA.
    \medskip
    
    R�cup�rer l'archive compress�e {\tt TP8.tar.gz} et extraire les fichiers qui sont contenus dans cette archive.
    
  	Se placer dans le r�pertoire {\tt TP8}:
    \begin{script}
        cd TP8
    \end{script}

    et pr�parer les compilations dans les points suivants avec la commande ci-dessous:
    \begin{script}
        ./build.sh
    \end{script}
	
\section{Outils logiciels}

Dans cette s�ance, on testera plusieurs outils qui permettent de faire de la programmation parall�le � un niveau plus �lev�.
Ces outils proposent des fonctions 
\begin{itemize}
	\item qui simplifient la configuration et le lancement des calculs parall�les,
	\item qui am�liorent le placement des donn�es,
	\item qui codent des algorithmes parall�les standards.
\end{itemize}

En g�n�ral, les performances obtenues ne sont pas les meilleures possibles (qu'on peut obtenir avec une programmation "manuelle" tr�s soign�e). Mais, ils repr�sentent un bon compromis entre facilit� de d�veloppement et niveau de performance.

Il est donc conseill� de tester ces outils avant de se lancer dans une progammation bas niveau.

\section{Elements de C++, niveau 11}

Plusieurs de ces outils utilisent des fonctionnalit�s C++11 (foncteurs, lambda-fonctions, typage automatique).

\begin{question}
Dans le r�pertoire {\tt src/C++11}., examiner les fichiers sources {\tt main1.cxx}, {\tt main2.cxx}, {\tt main3.cxx}, {\tt main4.cxx}, {\tt main5.cxx} et {\tt main6.cxx}
qui utilisent ces fonctionnalit�s.

Ex�cuter les programmes compil�s � partir de ces fichiers. Interpr�ter les r�sultats.
\end{question}

\section{Librairie TBB}


   


\end{document}
