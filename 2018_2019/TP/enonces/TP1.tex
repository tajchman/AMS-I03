\documentclass[12pt,a4paper]{article}
\usepackage[latin1]{inputenc}
\usepackage[french]{babel}
\usepackage{hyperref}
\usepackage{amsmath}
\usepackage{amsfonts}
\usepackage{amssymb}
\usepackage{graphicx}
\usepackage{layout}
\newenvironment{script}%
{\vspace{-15pt}\begin{quotation}\obeylines\noindent\ignorespaces\tt}%
{\par\noindent\ignorespacesafterend\end{quotation}}

\newenvironment{question}%
{\vspace{-15pt}{\it Question}\begin{quotation}\obeylines\noindent\ignorespaces}%
    {\par\noindent\ignorespacesafterend\end{quotation}}

\title{TP 1. Optimisation s�quentielle}
\date{}
\author{}

\parindent=0pt
\oddsidemargin=0pt
\marginparwidth=0pt
\textwidth=460pt
\textheight=650pt


\begin{document}
    %\layout
    \maketitle
    \section*{Conseil}
    
    Dans les s�ances du cours I03 du master AMS, on fournira des fichiers � utiliser comme support de cours ou de TP.
    
    On vous conseille de cr�er un r�pertoire vide dans votre espace de travail o� vous copierez ces fichiers et o� vous travaillerez dans le cadre de ce cours.
    
    Ceci afin d'�viter de m�langer les fichiers de ce cours avec ceux utilis�s lors d'autres enseignements.

	\section*{Pr�paration}
    
    R�cup�rer l'archive compress�e {\tt TP1.tar.gz} et extraire les fichiers qui sont contenus dans cette archive:
    \begin{script}
        cd <repertoire dans votre espace de travail>
        cp /home/t/tajchman/AMSI03/2018-11-30/TP1.tar.gz .
        tar xvfz TP1.tar.gz
    \end{script}

    Se placer dans le r�pertoire {\tt TP1}:
    \begin{script}
        cd TP1
    \end{script}

    et compiler avec la commande ci-dessous:
    \begin{script}
        ./build.sh
    \end{script}
	
    Remarque
    \begin{quotation}
        {\tt build.sh} est un fichier de commandes unix (dans le r�pertoire {\tt TP1}) qui contient les commandes pour compiler les codes dans plusieurs configurations. N'hesitez pas � examiner ce fichier et � poser des questions dessus.
    \end{quotation}

	\section{Outils de mesure du temps calcul}
    Il existe de nombreux moyen de mesurer le temps d'ex�cution de code ou de parties de code:
    \begin{itemize}
        \setlength{\itemsep}{5pt}
        \item[\textbullet] commande unix {\tt time} : mesure globale (temps ressenti par l'utilisateur) 
        
        \item[\textbullet] fonctions d�finies par le langage et utilisables depuis l'int�rieur du code : 
        
        \begin{quote}
            {\tt time = second()} (fortran), {\tt gettimeofday(...)} (C/C++), {\tt tic/toc} (matlab), etc.
        \end{quote}
        
        Permet de mesurer le temps d'ex�cution d'un groupe d'instructions.
        
        Penser � v�rifier dans la documentation quelle est la pr�cision des mesures.
        
        \item[\textbullet] Librairies, par exemple PAPI
        
         (\url{https://icl.cs.utk.edu/projects/papi/wiki/Main_Page})
        
        Permet de consulter des compteurs syst�me tr�s bas niveau (par exemple : nombre d'op�rations flottantes, utilisation des caches, utilisation des registres, etc.)
        
        \item[\textbullet] Outils externes de ``profilage'', ajoutent automatiquement des points de mesure dans le code (gprof), s'interposent entre le code et le syst�me pour r�cuperer des informations (valgrind, perf)
        
        \begin{quote}
            par exemple: {\tt gprof}, {\tt perf}, {\\ callgrind (valgrind)} (outils sous unix/linux), vtune (intel), etc.
        \end{quote}
        
        Permet de conna�tre des informations interm�diaires : nombre d'appels et temps moyen d'ex�cution de fonctions par exemple.
        
    \end{itemize}

    \medskip
    Les outils de mesure perturbent les temps de calcul, il donnent seulement une indication sur l'efficacit� d'un code et parfois quels sont les endroits du code les plus int�ressants � optimiser.
    
	
\end{document}