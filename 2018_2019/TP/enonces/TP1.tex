\documentclass[12pt,a4paper]{article}
\usepackage[latin1]{inputenc}
\usepackage[french]{babel}
\usepackage{amsmath}
\usepackage{amsfonts}
\usepackage{amssymb}
\usepackage{graphicx}

\newenvironment{script}%
{\vspace{-15pt}\begin{quotation}\obeylines\noindent\ignorespaces\tt}%
{\par\noindent\ignorespacesafterend\end{quotation}}

\title{TP 1. Optimisation s�quentielle}
\date{}

\parindent=0pt

\begin{document}
	
\maketitle
\section*{Conseil}

Dans les s�ances du cours I03 du master AMS, on fournira des fichiers � utiliser comme support de cours ou de TP.

On conseille de cr�er un r�pertoire vide dans votre espace de travail o� vous copierez ces fichiers et o� vous travaillerez dans le cadre de ce cours.

Ceci afin d'�viter de m�langer les fichiers de ce cours avec ceux utilis�s lors d'autres enseignements.
 
\section*{Pr�paration}
R�cup�rer l'archive compress�e {\tt TP1.tar.gz} et extraire les fichiers qui sont contenus dans cette archive:

\begin{script}
cd <repertoire dans votre espace de travail>
cp /home/t/tajchman/AMSI03/2018-11-30/TP1.tar.gz .
tar xvfz TP1.tar.gz
\end{script}

Se placer dans le r�pertoire {\tt TP1}:

\begin{script}
cd TP1
\end{script}

et compiler avec les commandes ci-dessous:

\begin{script}
mkdir build
cd build
cmake ../src
make 
\end{script}
	
	\section{Outils de mesure du temps calcul}
		Outil de mesure du temps calcul
	Localit� spatiale et temporelle
	
	
\end{document}