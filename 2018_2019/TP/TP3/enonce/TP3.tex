\documentclass[12pt,a4paper]{article}
\usepackage[T1]{fontenc}
%\usepackage[utf8]{inputenc}
\usepackage[latin1]{inputenc}

\usepackage[french]{babel}
\frenchbsetup{StandardLists=true}
\usepackage{enumitem}

\usepackage{hyperref}
\usepackage{amsmath}
\usepackage{amsfonts}
\usepackage{amssymb}
\usepackage{graphicx}
\usepackage{layout}
\newenvironment{script}%
{\vspace{-22pt}\begin{quote}\obeylines\noindent\ignorespaces\tt}%
{\par\noindent\ignorespacesafterend\end{quote}}

\newcounter{question}
\newenvironment{question}[1][]%
{\vspace{5pt}\refstepcounter{question}\vspace{2pt}\textbf{Question \thequestion. #1}%
	\begin{quote}\obeylines\ignorespaces}%
	{\par\noindent\ignorespacesafterend\end{quote}}

\newenvironment{remarque}%
{\vspace{2pt}\textit{Remarque}\begin{quote}\obeylines\ignorespaces}%
	{\par\noindent\ignorespacesafterend\end{quote}}

\newenvironment{commandes}%
{\vspace{2pt}\textbf{Commandes}\begin{quote}\obeylines\ignorespaces}%
	{\par\noindent\ignorespacesafterend\end{quote}}

\newenvironment{liste}%
{\begin{itemize}[parsep=1pt]}%
{\end{itemize}}

\parindent=0pt
\oddsidemargin=0pt
\marginparwidth=0pt
\textwidth=460pt
\textheight=650pt
\headsep=0pt


\begin{document}
%    \layout
    
    \begin{center}
    	\LARGE TP 3. Programmation multi-threads (2)
    \end{center}

	\section*{Pr�paration}
    
    R�cup�rer l'archive compress�e {\tt TP3.tar.gz} et extraire les fichiers qui sont contenus dans cette archive:
    \begin{script}
        cd <repertoire dans votre espace de travail>
        cp /home/t/tajchman/AMSI03/2018-12-14/TP3.tar.gz .
        tar xvfz TP3.tar.gz
    \end{script}

    Se placer dans le r�pertoire {\tt TP3}:
    \begin{script}
        cd TP3
    \end{script}

    et pr�parer les compilations dans les points suivants avec les commandes ci-dessous:
    \begin{script}
        mkdir -p build
        cd build
        cmake ../src
        make install
        cd ..
    \end{script}
	

\section{Exemple OpenMP ``grain fin''}

Le fichier {\tt src/sinus\_fine/sinus.cxx} contient la version OpenMP ``grain fin'' obtenue � la fin du TP pr�c�dent.

On s'en servira de base pour les versions dites ``grain grossier''.

\begin{question}
	Comparer les temps d'ex�cution de la version s�quentielle et de cette version en tapant les commandes
    \begin{script}
		./build/sinus\_seq/sinus\_seq 40000
		OMP\_NUM\_THREADS=3 ./build/sinus\_fine/sinus\_fine 40000
	\end{script}
\end{question}

\section{Exemple OpenMP ``grain grossier''}

Le fichier {\tt src/sinus\_coarse\_1/sinus.cxx} contient une version OpenMP ``grain grossier''.

\begin{question}
	Examiner ce fichier et comparez-le � la version ``grain fin''.
	
	Comparer les temps d'ex�cution de la version ``grain fin''et de cette version en tapant les commandes
	\begin{script}
		OMP\_NUM\_THREADS=3 \textbackslash
		time -p ./build/sinus\_fine/sinus\_fine 40000
		OMP\_NUM\_THREADS=3 \textbackslash
		time -p ./build/sinus\_coarse\_1/sinus\_coarse\_1 40000
	\end{script}

	Il y a peu de diff�rence (normalement) entre les deux versions. Expliquer pourquoi.
	
	Le code affiche aussi les temps de calcul des diff�rents threads.
\end{question}

\section{Exemple OpenMP ``grain grossier'' avec �quilibrage de charge}

Le calcul du sinus en utilisant un d�veloppement de Taylor a �t� volontairement ralenti pour accentuer la diff�rence de temps calcul de $x \mapsto \sin x$ pour diff�rentes valeurs de $x$.

Il s'en suit que les threads ne prennent pas le m�me temps de calcul suivant la plage des valeurs de $x$ qui leur sont attribu�e (et qui est la m�me que dans le cas ``grain fin"), voir le fichier {\tt src/sin.cxx}.

Dans cette version, on utilise un algorithme d'�quilibrage de charge entre les diff�rents threads.

\begin{question}
	Examiner le fichier {\tt src/sinus\_coarse\_2/charge.cxx} qui contient cet algorithme et le fichier {\tt src/sinus\_coarse\_2/sinus.cxx} qui l'utilise.
	
	Ex�cuter plusieurs fois la commande
	\begin{script}
		OMP\_NUM\_THREADS=3 \textbackslash
		time ./build/sinus\_coarse\_2/sinus\_coarse\_2 40000
	\end{script}
	
	Chaque ex�cution tente d'am�liorer les temps calcul en adaptant la r�partition de charge de mieux en mieux (si possible).
\end{question}

\begin{remarque}
L'algorithme d'�quilibrage de charge utilis� ici n'est pas optimal. Vous �tes encourag�s � l'�tudier et � l'am�liorer.
\end{remarque}

\vfill\eject
\section{Parall�lisation du (mini-)code avec le mod�le OpenMP ``grain grossier''}

Le r�pertoire {code/PoissonOpenMP\_CoarseGrain} contient la version du code parall�lis� par de l'OpenMP ``grain fin''.

\begin{question}
Remplacer la parall�lisation ``grain fin'' par le mod�le ``gros grain'' (sans perte de charge).
\end{question}

\section{Parall�lisation en utilisant le concept de t�ches OpenMP}

\end{document}
