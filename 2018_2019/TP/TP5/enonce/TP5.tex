\documentclass[12pt,a4paper]{article}
\usepackage[T1]{fontenc}
%\usepackage[utf8]{inputenc}
\usepackage[latin1]{inputenc}

\usepackage[french]{babel}
\frenchbsetup{StandardLists=true}
\usepackage{enumitem}

\usepackage{hyperref}
\usepackage{amsmath}
\usepackage{amsfonts}
\usepackage{amssymb}
\usepackage{graphicx}
\usepackage{layout}
\newenvironment{script}%
{\vspace{-22pt}\begin{quote}\obeylines\noindent\ignorespaces\tt}%
{\par\noindent\ignorespacesafterend\end{quote}}

\newcounter{question}
\newenvironment{question}[1][]%
{\vspace{5pt}\refstepcounter{question}\vspace{2pt}\textbf{Question \thequestion. #1}%
	\begin{quote}\obeylines\ignorespaces}%
	{\par\noindent\ignorespacesafterend\end{quote}}

\newenvironment{remarque}%
{\vspace{2pt}\textit{Remarque}\begin{quote}\obeylines\ignorespaces}%
	{\par\noindent\ignorespacesafterend\end{quote}}

\newenvironment{commandes}%
{\vspace{2pt}\textbf{Commandes}\begin{quote}\obeylines\ignorespaces}%
	{\par\noindent\ignorespacesafterend\end{quote}}

\newenvironment{liste}%
{\begin{itemize}[parsep=1pt]}%
{\end{itemize}}

\parindent=0pt
\oddsidemargin=0pt
\marginparwidth=0pt
\textwidth=460pt
\textheight=650pt
\headsep=0pt


\begin{document}
%    \layout
    
    \begin{center}
    	\LARGE TP 5. Programmation hybride MPI - OpenMP (2)
    \end{center}

	\section*{Pr�paration}

R�cup�rer l'archive compress�e {\tt TP5.tar.gz} et extraire les fichiers qui sont contenus dans cette archive:
\begin{script}
	cd <repertoire dans votre espace de travail>
	cp /home/t/tajchman/AMSI03/2019-01-11/TP5.tar.gz .
	tar xvfz TP5.tar.gz
\end{script}

Se placer dans le r�pertoire {\tt TP5}:
\begin{script}
	cd TP5
\end{script}

et pr�parer les compilations dans les points suivants avec la commande ci-dessous:
\begin{script}
	./build.sh
\end{script}

	\section{Parall�lisation des �changes de messages}
    
    Il est possible avec les sections OpenMP, d'ex�cuter plusieurs instructions par des threads diff�rents, par exemple :
    
    \begin{script}
\#pragma omp parallel sections
\{
\quad	\#pragma omp section
\quad	\{
\quad\quad		/* Execute par le thread 1 */
\quad	\} 
\quad	\#pragma omp section
\quad	\{
\quad\quad		/* Execute par le thread 2 */
\quad	\} 
\quad	\#pragma omp section
\quad	\{
\quad\quad		/* Execute par le thread 3 */
\quad	\} 
\quad	/* ... */
\}
    \end{script}
     
\begin{question}
	
On fournit une version du code {\tt src/PoissonMPI\_OpenMP\_FineGrain} (parall�lis� par MPI et par OpenMP - grain fin).

Utiliser des sections OpenMP dans le fichier {\tt value.cxx}, lignes 130 � 227, pour ex�cuter les �changes MPI dans plusieurs threads diff�rents.

Ne pas oublier de sp�cifier un niveau suffisant de compatibilit� OpenMP-MPI.
\end{question}

\section{Recouvrement communications/calculs}

Il est possible de faire du recouvrement communications/calculs en MPI seul, on propose de le faire ici avec une programmation hybride MPI-OpenMP.

\begin{question}
	On fournit une version {\tt src/PoissonMPI} parall�lis�e par MPI mais pr�te � accueillir des pragma OpenMP.
	
	Pour chaque it�ration en temps, s�parer le calcul des inconnues en deux parties:
	\begin{enumerate}
		\item une premi�re partie qui ne d�pend pas des valeurs transmises par des �changes  MPI (au cours de la m�me it�ration)
		\item une seconde partie qui en d�pend
	\end{enumerate}
	
\end{question}

Le but est de calculer les inconnues qui ne d�pendent pas des �changes MPI en m�me temps que ces �changes.

\begin{question}
	Faire ex�cuter dans deux threads diff�rents :
	
	\begin{itemize}
		\item les communications MPI et les calculs qui en d�pendent,
		\item les autres calculs.
	\end{itemize}
	
	Ne pas oublier de sp�cifier un niveau suffisant de compatibilit� OpenMP-MPI.
\end{question}

 
\end{document}
