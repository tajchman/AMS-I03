\documentclass[12pt,a4paper]{article}
\usepackage[latin1]{inputenc}
\usepackage[francais]{babel}
\usepackage{amsmath}
\usepackage{amsfonts}
\usepackage{amssymb}
\usepackage{graphicx}
\title{TP 1. Optimisation s�quentielle}
\date{}
\begin{document}
	
\maketitle
\section*{Pr�paration}
R�cup�rer l'archive compress�e {\tt TP1.tar.gz} et extraire les fichiers qui sont contenus dans cette archive.
\begin{verbatim}
cd <repertoire dans votre espace de travail>
cp /home/t/tajchman/AMS_I03/TP1.tar.gz .
tar xvfz TP1.tar.gz
\end{verbatim}
	
Compiler les codes contenus dans TP1 et essayez-les.	\begin{verbatim}
	cp /home/t/tajchman/TP1.tar.gz 
	\end{verbatim}
	
	Extraire les fichiers et compiler les codes qui sont contenus dans l'archive :tar.gz}
	\begin{verbatim}
	tar xvfz TP1.tar.gz
	\end{verbatim}
	
	\section{Outils de mesure du temps calcul}
		Outil de mesure du temps calcul
	Localit� spatiale et temporelle
	
	
\end{document}