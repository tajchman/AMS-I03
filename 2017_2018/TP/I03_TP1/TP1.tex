\documentclass{beamer}
\usepackage[utf8]{inputenc}
\usetheme[]{boxes}
\usecolortheme{seagull}
	
\begin{document}
\begin{frame}
	\frametitle{TP 1 : Pr\'eparation}
	
	\vfill
	\textcolor{blue}{\bf A chaque \'etape, regarder les messages affich\'es pour voir si tout s'est bien pass\'e !}
	\vfill

	\begin{enumerate}
		\item R\'ecup\'erer l'archive {\tt TP1.tar.gz} et extraire les fichiers.
		\item Ouvrir un terminal et se placer dans le r\'epertoire {\tt I03\_TP1} qui vient d'\^etre cr\'e\'e
		\item préparer la compilation du code du TP avec les commandes :
		\begin{quote}
			mkdir -p build\\
			cd build\\
			cmake ../src\\
			cd ..
		\end{quote}
		\item Se remettre dans le r\'epertoire {\tt I03\_TP1} et compiler:
		\begin{quote}
			make -C build
		\end{quote}
	\end{enumerate}
	\vfill
	
\end{frame}
\begin{frame}
	\begin{enumerate}
  		\setcounter{enumi}{4}
		\item Executer le code avec la commande:
		\begin{quote}
			./build/PoissonSeq
		\end{quote}
		\item A la fin de l'exécution, les résultats sont sauvegardés au format VTK dans un répertoire "results\_\ldots" (le nom précis est affiché à l'écran)
	\end{enumerate}

\vfill
\textcolor{blue}{\bf Si on modifie un ou plusieurs fichiers sources (dans le sous-r\'epertoire src), il faut recompiler (point 4).}
\vfill

\textcolor{blue}{\bf Si on ajoute un nouveau fichier ou on enl\`eve un fichier existant (dans le sous-r\'epertoire src), il faut adapter les fichiers CMakeLists.txt et refaire les points 3 et 4.}
\vfill

\end{frame}

\begin{frame}[fragile]
	\frametitle{Mesure du temps de calcul global}
    \vfill
	Afficher le temps de calcul global avec {\tt time} :
		\begin{quote}
	       time ./build/PoissonSeq
        \end{quote}
	
	A l'\'ecran:\begin{minipage}[t]{4cm}
	\begin{verbatim}
	   real    0m30,283s
	   user    0m30,186s
	   sys     0m0,096s
	\end{verbatim}
	\end{minipage}

    \vfill
	\begin{itemize}
		\item Avantage : n'est pas intrusif
		\begin{quote}
			pas besoin de modifier le code, ni de le compiler avec des options sp\'ecifiques.
		\end{quote} 
		\item D\'esavantage : donne une information globale
		\begin{quote}
			on ne sait pas dans quelle partie du code, on passe peu/beaucoup de temps, ni pourquoi.
		\end{quote} 
	\end{itemize}
    \vfill
\end{frame}

\begin{frame}
	\frametitle{Mesure plus pr\'ecise : utilisation d'outils de ``profiling''}
    
    Exemple : {\tt gprof}
    
	\vfill
	Fait partie de la famille gcc/g++/gfortran.
	
	
	\begin{itemize}
	\item \textcolor{blue}{Avantage} : calcule le nombre d'appels de chaque fonction et le (pourcentage du) temps qui y est pass\'e
	\item \textcolor{blue}{Avantage} : n'est pas tr\`es intrusif (pas besoin de modifier le code, mais il faut le recompiler avec une option sp\'ecifique: -pg).
	\item \textcolor{red}{D\'esavantage} : le temps pass\'e dans une fonction est peu pr\'ecis dans une fonction ``courte"
	\item \textcolor{red}{D\'esavantage} : ne mesure pas le temps dans les diff\'erentes parties d'une fonction.
	\item \textcolor{red}{D\'esavantage} : ne rentre pas dans les librairies dynamiques.
\end{itemize}
	\vfill
\end{frame}
\begin{frame}[fragile]
	
{\bf 	Mode de fonctionnement:}
	\begin{quote}
		Ajoute dans chaque fonction, un comptage du nombre d'appels de cette fonction et \'evalue statistiquement le temps pass\'e dans cette fonction (tous les 0.01 secondes on enregistre dans quelle fonction on se trouve).
	\end{quote} 

	\vfill
{\bf Utilisation de {\tt gprof}}
	\vfill

Recompiler en utilisant l'option -pg:
	\begin{quote}
		\begin{verbatim}
			mkdir -p build_gprof
			cd build_gprof
			cmake -DCMAKE_CXX_FLAGS="-pg" ../src
			cd ..
			make -C build_gprof
		\end{verbatim}
	\end{quote} 

Exécuter le code:
		\begin{quote}
		\begin{verbatim}
			./build_gprof/PoissonSeq
		\end{verbatim}
		\end{quote}
	
Collecter les mesures
		\begin{quote}
		\begin{verbatim}
			gprof ./build_gprof/PoissonSeq
		\end{verbatim}
		\end{quote}

\end{frame}

\begin{frame}
	\frametitle{Mesure plus pr\'ecise : utilisation d'outils de ``profiling''}
	
	Exemple : {\tt perf}
	
	\vfill
	Outil spécifique linux
	
	\begin{itemize}
		\item \textcolor{blue}{Avantage} : tr\`es puissant (mesure le temps pass\'e dans une fonction, une instruction C/C++/fortran, une instruction binaire, multiples indicateurs de performance)
		\item \textcolor{blue}{Avantage} : non intrusif
		\item \textcolor{red}{D\'esavantage} : plus compliqu\'e \`a utiliser
		\item \textcolor{red}{D\'esavantage} : ne rentre pas toujours dans les librairies dynamiques.
		\item \textcolor{red}{D\'esavantage} : n\'ecessite que la machine soit configur\'ee  correctement.
	\end{itemize}
	\vfill
	\textcolor{blue}{\bf Outil \'a privil\'egier quand c'est possible}
\end{frame}
\begin{frame}[fragile]
	
	{\bf 	Mode de fonctionnement:}
	\begin{quote}
		Enregistre les \'ev\'enements dans le noyau Linux, \'evalue statistiquement le temps pass\'e dans les fonctions et les instructions.
	\end{quote} 
	
	\vfill
	{\bf Utilisation simple de {\tt perf}}
	\vfill
	
	Recompiler en utilisant l'option -pg:
	\begin{quote}
		\begin{verbatim}
		mkdir -p build_gprof
		cd build_gprof
		cmake -DCMAKE_CXX_FLAGS="-pg" ../src
		cd ..
		make -C build_gprof
		\end{verbatim}
	\end{quote} 
	
	Exécuter le code:
	\begin{quote}
		\begin{verbatim}
		./build_gprof/PoissonSeq
		\end{verbatim}
	\end{quote}
	
	Collecter les mesures
	\begin{quote}
		\begin{verbatim}
		gprof ./build_gprof/PoissonSeq
		\end{verbatim}
	\end{quote}
	
\end{frame}

\end{document}