\documentclass{beamer}
\usepackage[utf8]{inputenc}
\usetheme[]{boxes}
\usecolortheme{seagull}
	
\begin{document}
\begin{frame}
	\frametitle{TP 1 : mesures des temps calcul (cas non parall\`ele)}
	
	Pr\'eparation du TP:
	\vfill
	A chaque \'etape, regarder les messages affich\'es pour voir si tout s'est bien pass\'e !
	\begin{enumerate}
		\item R\'ecup\'erer l'archive {\tt TP1.tar.gz} et extraire les fichiers.
		\item Ouvrir un terminal et se placer dans le r\'epertoire {\tt I03\_TP1} qui vient d'\^etre cr\'e\'e
		\item préparer la compilation du code du TP avec les commandes :
		\begin{quote}
			mkdir -p build\\
			cd build\\
			cmake ../src\\
			cd ..
		\end{quote}
		\item Se remettre dans le r\'epertoire {\tt I03\_TP1} et compiler:
		\begin{quote}
			make -C build
		\end{quote}
	\end{enumerate}
	
\end{frame}
\begin{frame}
	\begin{enumerate}
  		\setcounter{enumi}{4}
		\item Executer le code avec la commande:
		\begin{quote}
			./build/PoissonSeq
		\end{quote}
		\item A la fin de l'exécution, les résultats sont sauvegardés au format VTK dans un répertoire "results\_n\ldots" (le nom précis est affiché à l'écran)

		\item Si on modifie un ou plusieurs fichiers sources (dans le sous r\'epertoire {\tt 	src}), il faut recompiler (point 4).
	\end{enumerate}
\end{frame}
\begin{frame}
	\frametitle{Mesure du temps de calcul global}
	Afficher le temps de calcul global avec {\tt time}.
	
	\begin{itemize}
		\item Avantage : n'est pas intrusif (le code )
	\end{itemize}
\end{frame}
	
\end{document}