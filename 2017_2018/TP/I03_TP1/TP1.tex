\documentclass{beamer}
\usepackage[utf8]{inputenc}
\usetheme[]{boxes}
\usecolortheme{seagull}
	
\begin{document}
\begin{frame}
	\frametitle{TP 1 : Pr\'eparation}
	
	\vfill
	\textcolor{blue}{\bf A chaque \'etape, regarder les messages affich\'es pour voir si tout s'est bien pass\'e !}
	\vfill

	\begin{enumerate}
		\item R\'ecup\'erer l'archive {\tt TP1.tar.gz} et extraire les fichiers.
		\item Ouvrir un terminal et se placer dans le r\'epertoire {\tt I03\_TP1} qui vient d'\^etre cr\'e\'e
		\item préparer la compilation du code du TP avec les commandes :
		\begin{quote}
			mkdir -p build\\
			cd build\\
			cmake ../src\\
			cd ..
		\end{quote}
		\item Se remettre dans le r\'epertoire {\tt I03\_TP1} et compiler:
		\begin{quote}
			make -C build
		\end{quote}
	\end{enumerate}
	\vfill
	
\end{frame}
\begin{frame}
	\begin{enumerate}
  		\setcounter{enumi}{4}
		\item Executer le code avec la commande:
		\begin{quote}
			./build/PoissonSeq
		\end{quote}
		\item A la fin de l'exécution, les résultats sont sauvegardés au format VTK dans un répertoire "results\_\ldots" (le nom précis est affiché à l'écran)
	\end{enumerate}

\vfill
\textcolor{blue}{\bf Si on modifie un ou plusieurs fichiers sources (dans le sous-r\'epertoire src), il faut recompiler (point 4).}
\vfill

\textcolor{blue}{\bf Si on ajoute un nouveau fichier ou on enl\`eve un fichier existant (dans le sous-r\'epertoire src), il faut adapter les fichiers CMakeLists.txt et refaire les points 3 et 4.}
\vfill

\end{frame}

\begin{frame}[fragile]
	\frametitle{Mesure du temps de calcul global}
	Afficher le temps de calcul global avec {\tt time} :
		\begin{quote}
	       time ./build/PoissonSeq
        \end{quote}
	
	A l'\'ecran:
	\begin{verbatim}
	   real    0m30,283s
	   user    0m30,186s
	   sys     0m0,096s
	\end{verbatim}
	
	\begin{itemize}
		\item Avantage : n'est pas intrusif (pas besoin de modifier le code, ni de le compiler avec des options sp\'ecifiques)
		\begin{quote}
			pas besoin de modifier le code, ni de le compiler avec des options sp\'ecifiques.
		\end{quote} 
		\item D\'esavantage : donne une information globale
		\begin{quote}
			on ne sait pas dans quelle partie du code, on passe peu/beaucoup de temps, ni pourquoi.
		\end{quote} 
	\end{itemize}
\vfill
\end{frame}
	
\end{document}