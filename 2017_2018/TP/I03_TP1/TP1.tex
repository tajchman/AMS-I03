\documentclass{beamer}
\usepackage[utf8]{inputenc}
\usetheme[]{boxes}
\usecolortheme{seagull}
\usepackage{listings}

\begin{document}
\begin{frame}
	\frametitle{TP 1 : Contexte}
	
	Le code propos\'e dans ce TP calcule une solution approch\'ee en 3 dimensions d'espace $(x,y,z)$ de l'\'equation de la chaleur dans un cube $[0,1]^3$ :
	$$
	\frac{\partial u}{\partial t} = \Delta u
	$$
	avec conditions aux limites de Dirichlet au bord.
	\vfill
	
	Le calcul se fait a l'aide d'une m\'ethode explicite en temps (Euler) et discrt\'etisation par diff\'erences finies en espace.
	
	\vfill
	Dans la derniere partie du TP, on peut ajouter un terme d'advection (linéaire) $\frac{\partial u}{\partial x}$.
	
	\vfill
	
\end{frame}

\begin{frame}
\frametitle{TP 1 : Pr\'eparation}
	
	\vfill
	\textcolor{blue}{\bf A chaque \'etape, regarder les messages affich\'es pour voir si tout s'est bien pass\'e !}
	\vfill

	\begin{enumerate}
		\item R\'ecup\'erer l'archive {\tt TP1.tar.gz} et extraire les fichiers.
		\item Ouvrir un terminal et se placer dans le r\'epertoire {\tt I03\_TP1} qui vient d'\^etre cr\'e\'e
		\item préparer la compilation du code du TP avec les commandes :
		\begin{quote}
			mkdir -p build\\
			cd build\\
			cmake ../src\\
			cd ..
		\end{quote}
		\item Se remettre dans le r\'epertoire {\tt I03\_TP1} et compiler:
		\begin{quote}
			make -C build
		\end{quote}
	\end{enumerate}
	\vfill
	
\end{frame}
\begin{frame}
	\begin{enumerate}
  		\setcounter{enumi}{4}
		\item Executer le code avec la commande:
		\begin{quote}
			./build/PoissonSeq
		\end{quote}
		\item A la fin de l'exécution, les résultats sont sauvegardés au format VTK dans un répertoire "results\_\ldots" (le nom précis est affiché à l'écran)
	\end{enumerate}

\vfill
\textcolor{blue}{\bf Si on modifie un ou plusieurs fichiers sources (dans le sous-r\'epertoire src), il faut recompiler (point 4).}
\vfill

\textcolor{blue}{\bf Si on ajoute un nouveau fichier ou on enl\`eve un fichier existant (dans le sous-r\'epertoire src), il faut adapter les fichiers CMakeLists.txt et refaire les points 3 et 4.}
\vfill

\end{frame}

\begin{frame}[fragile]
\vfill
Les commandes ci-dessus g\'enerent une version optimis\'ee par la compilateur (``Release''). Si n\'ecessaire, on peut compiler une version ``Debug'' (non optimis\'ee)
qui permet d'utiliser un outil de d\'ebug (ex\'ecution pas \`a pas, afficher des valeurs en cours de calcul, etc).
\vfill

Certains des outils donnent aussi plus de renseignements sur une version ``Debug'' que sur une version ``Release''.
\vfill

Remplacer les commandes de 3. et 4. par:
	\begin{quote}
	\begin{verbatim}
	mkdir -p build_debug
	cd build_debug
	cmake -DCMAKE_BUILD_TYPE=Debug ../src
	cd ..
	make -C build_debug
	\end{verbatim}
\end{quote}
\vfill

\end{frame}

\begin{frame}[fragile]
	\frametitle{Mesure du temps de calcul global}
    \vfill
	Afficher le temps de calcul global avec {\tt time} :
		\begin{quote}
	       time ./build/PoissonSeq
        \end{quote}
	
	A l'\'ecran (exemple) :\begin{minipage}[t]{4cm}
	\begin{verbatim}
	   real    0m30,283s
	   user    0m30,186s
	   sys     0m0,096s
	\end{verbatim}
	\end{minipage}

    \vfill
	\begin{itemize}
		\item Avantage : n'est pas intrusif
		\begin{quote}
			pas besoin de modifier le code, ni de le compiler avec des options sp\'ecifiques.
		\end{quote} 
		\item D\'esavantage : donne une information globale
		\begin{quote}
			on ne sait pas dans quelle partie du code, on passe peu/beaucoup de temps, ni pourquoi.
		\end{quote} 
	\end{itemize}
    \vfill
\end{frame}

\begin{frame}
	\frametitle{Mesure plus pr\'ecise : outil de ``profiling''{\tt gprof}}
    
	\vfill
	Fait partie de la famille gcc/g++/gfortran.
	
	
	\begin{itemize}
	\item \textcolor{blue}{Avantage} : calcule le nombre d'appels de chaque fonction et le (pourcentage du) temps qui y est pass\'e
	\item \textcolor{blue}{Avantage} : n'est pas tr\`es intrusif (pas besoin de modifier le code, mais il faut le recompiler avec une option sp\'ecifique: -pg).
	\item \textcolor{red}{D\'esavantage} : le temps pass\'e dans une fonction est peu pr\'ecis dans une fonction ``courte"
	\item \textcolor{red}{D\'esavantage} : ne mesure pas le temps dans les diff\'erentes parties d'une fonction.
	\item \textcolor{red}{D\'esavantage} : ne rentre pas dans les librairies dynamiques.
\end{itemize}
	\vfill
\end{frame}
\begin{frame}[fragile]
	
{\bf 	Mode de fonctionnement:}
	\begin{quote}
		Ajoute dans chaque fonction, un comptage du nombre d'appels de cette fonction et \'evalue statistiquement le temps pass\'e dans cette fonction (tous les 0.01 secondes on enregistre dans quelle fonction on se trouve).
	\end{quote} 

	\vfill
{\bf Utilisation de {\tt gprof}}
	\vfill

Recompiler en utilisant l'option -pg:
	\begin{quote}
		\begin{verbatim}
			mkdir -p build_gprof
			cd build_gprof
			cmake -DCMAKE_CXX_FLAGS="-pg" ../src
			cd ..
			make -C build_gprof
		\end{verbatim}
	\end{quote} 

Exécuter le code:
		\begin{quote}
		\begin{verbatim}
			./build_gprof/PoissonSeq
		\end{verbatim}
		\end{quote}
	
Collecter les mesures
		\begin{quote}
		\begin{verbatim}
			gprof ./build_gprof/PoissonSeq
		\end{verbatim}
		\end{quote}

\end{frame}

\begin{frame}
	\frametitle{Mesure plus pr\'ecise : outil de ``profiling'' {\tt perf}}
		
	\vfill
	Outil spécifique linux
	
	\begin{itemize}
		\item \textcolor{blue}{Avantage} : tr\`es puissant (mesure le temps pass\'e dans une fonction, une instruction C/C++/fortran, une instruction binaire, multiples indicateurs de performance)
		\item \textcolor{blue}{Avantage} : non intrusif
		\item \textcolor{red}{D\'esavantage} : plus compliqu\'e \`a utiliser
		\item \textcolor{red}{D\'esavantage} : ne rentre pas toujours dans les librairies dynamiques.
		\item \textcolor{red}{D\'esavantage} : n\'ecessite que la machine soit configur\'ee  correctement.
	\end{itemize}
	\vfill
	\textcolor{blue}{\bf Outil \'a privil\'egier quand c'est possible}
\end{frame}
\begin{frame}[fragile]
	
	{\bf 	Mode de fonctionnement:}
	\begin{quote}
		Enregistre les \'ev\'enements dans le noyau Linux, \'evalue statistiquement le temps pass\'e dans les fonctions et les instructions.
	\end{quote} 
	
	\vfill
	{\bf Utilisation simple de {\tt perf}}
	\vfill
	
	Il n'est pas toujours n\'ecessaire de recompiler. Sur certaines machines, une version ``Debug" est pr\'ef\'erable. 
	
	\vfill
	Instrumenter le code (g\'en\'erer les mesures):
	\begin{quote}
		\begin{verbatim}
		perf record ./build/PoissonSeq
		(ou perf record ./build_debug/PoissonSeq)
		perf report
		\end{verbatim}
	\end{quote} 
	
	\vfill
	Voir la documentation de perf pour les (nombreuses) autres options.
	
	\vfill
\end{frame}

\begin{frame}
\frametitle{Mesure plus pr\'ecise : outil de ``profiling'' {\tt valgrind-callgrind}}

\begin{itemize}
	\item \textcolor{blue}{Avantage} : interm\'ediaire (mesure le temps pass\'e dans une fonction, une instruction C/C++/fortran)
	\item \textcolor{blue}{Avantage} : non intrusif
	\item \textcolor{blue}{Avantage} : g\`ere mieux les librairies dynamiques.
	\item \textcolor{blue}{Avantage} : l'outil {\tt kcachegrind} offre une interface graphique pratique \`a utiliser.
	\item \textcolor{red}{D\'esavantage} : temps d'ex\'ecution multipli\'e par $\approx$ 40
\end{itemize}
\vfill
\end{frame}

\begin{frame}[fragile]
Mode de fonctionnement:
\begin{quote}
	Le code s'ex\'ecute dans une machine virtuelle (simule une machine ``id\'eale''). Permet de mesurer pr\'ecisement un grand nombre d'indicateurs. Mais explique le facteur de ralentissement.
\end{quote}

{\bf Utilisation:}
Utiliser une version ``Debug''.

\vfill
Instrumenter le code
	\begin{quote}
		\begin{verbatim}
		valgrind --tool=callgrind ./build_debug/PoissonSeq
		\end{verbatim}
	\end{quote}

Cr\'ee un fichier {\tt callgrind.out.XXXX} o\`u {\tt XXXX} est le num\'ero de processus qu'on vient d'ex\'ecuter.

\vfill
Afficher et explorer les r\'esultats de mesure:
\begin{quote}
	\begin{verbatim}
	kcachegrind callgrind.out.XXXX
	\end{verbatim}
\end{quote}
\vfill

\end{frame}

\begin{frame}[fragile]
\frametitle{Mesure ``manuelle'' des temps de calcul.}

Principe: 
\begin{itemize}
	\item encadrer le bloc d'instructions que l'on veut mesurer par des appels \`a des fonctions qui renvoient la valeur de l'horloge interne de la machine,
	\item calculer le temps d'ex\'ecution dans ce bloc en faisant la diff\'erence des valeurs ci-dessus.
\end{itemize}
\vfill

Fonctions syst\`eme disponibles:
\begin{itemize}
	\item \verb clock() \quad(fonction syst\`eme C)
	\item \verb gettimeofday(...) \quad(fonction syst\`eme C)
	\item \verb system_clock(...) \quad(fonction syst\`eme fortran)
	\item \verb high_resolution_clock \quad(classe C++ 11)
	\item ...
\end{itemize}
\vfill
\end{frame}


\begin{frame}
Librairies externes (timers haute r\'esolution) :
\begin{itemize}
	\item PAPI : \url{http://icl.cs.utk.edu/papi}
	\item BoostTimers : \url{http://www.boost.org/doc/libs/1_65_1/libs/timer/doc/cpu_timers.html}
\end{itemize}
\vfill

Remarques
\begin{quote}
	Chaque fonction syst\`eme est utilisable ou non suivant le langage de programmation utilis\'e.
	
	La pr\'ecision n'est pas la m\^eme: consulter la documentation.
\end{quote}
\vfill
\end{frame}


\begin{frame}[fragile]

Exemple avec la fonction C syst\`eme {\tt clock}:

\vfill
\begin{lstlisting}
#include<stdio.h>   
#include<math.h>   
#include<time.h>   
int main()   
{   
  clock_t t1, t2;
  t1 = clock();   
	
  int n=100000;
  double s = 0.0;
  for(int i = 0; i < n; i++) s += sin((0.2*i)/n);  
	
  t2 = clock();   
  float diff = (float)(t2 - t1)/CLOCKS_PER_SEC;   
  printf("temps calcul : %f s",diff);   
  return 0;   
}
\end{lstlisting}

\end{frame}

\begin{frame}[fragile]
On fournit une classe C++ Timer
avec les fonctionnalités suivantes:

\begin{lstlisting}
 //definit une variable "chronometre"
Timer T; 
//demarrer le chronometre T
T.start();
//arreter le chronometre T
T.stop();
//remettre a zero
T.reset();

// retourne le temps mesure 
// entre un appel start() et un stop()
// (utiliser juste apres le stop()
double dt = T.elapsed(); 
\end{lstlisting}

\end{frame}

\begin{frame}[fragile]
	\vfill
	Se mettre dans le r\'epertoire {\tt I03\_TP1}, faire une copie du r\'epertoire {\tt src} dans {Q1/src} et compiler:
	\begin{quote}
		\begin{verbatim}
		mkdir -p Q1/build
		cp -rf src Q1/src
		cd Q1/build && cmake ../src && cd ..
		make -C build
		\end{verbatim}
	\end{quote}

	\vfill
	\textcolor{blue}{\bf Q1. Utiliser la classe Timer pour mesurer s\'epar\'ement}
	\begin{itemize}
		\item \textcolor{blue}{\bf le temps d'initialisation (lignes 31-32 du fichier {\tt src/main.cxx})}
		\item \textcolor{blue}{\bf la moyenne du temps d'ex\'ecution d'une it\'eration de calcul (lignes 43-45 du fichier {\tt src/main.cxx})}
	\end{itemize}

	\textcolor{blue}{\bf Executer le code avec et sans l'option {\tt out=10}}:
	\begin{quote}
		\color{blue}
		\begin{verbatim}
			./build/PoissonSeq
			./build/PoissonSeq out=10
		\end{verbatim}
	\end{quote}
    (la seconde ex\'ecution sauvegarde les r\'esultats sur fichier)
	

\vfill
\end{frame}

\begin{frame}[fragile]
	\vfill
	Se mettre dans le r\'epertoire {\tt I03\_TP1}, faire une copie du r\'epertoire {\tt Q1/src} dans {Q2/src} et compiler:
	\begin{quote}
		\begin{verbatim}
		mkdir -p Q2/build
		cp -rf Q1/src Q2/src
		cd Q2/build && cmake ../src && cd ..
		make -C build
		\end{verbatim}
	\end{quote}

	\vfill
	\textcolor{blue}{\bf Q2. La classe {Values} rep\'esente un ensemble de $n_0 \times n_1 \times n_2$ valeurs et poss\`ede des fonctions ({\tt operator()}) qui retournent la composante $v_{i,j,k}$ o\`u $v$ est de type {\tt Values}.}
	
	\begin{itemize}
		\item \textcolor{blue}{D\'ecrire comment sont rang\'ees en m\'emoire les composantes $v_{i,j,k}$
		(examiner les fichiers {\tt src/values.cxx} et {\tt src/values.hxx}).}
		\item \textcolor{blue}{Changer la fa\c{c}on de ranger les composantes en m\'emoire (il y a plusieurs possibilit\'es)}
		\item \textcolor{blue}{Comparer les r\'esultats et les temps de calcul avec la version dans {\tt Q1}.}
		\item \textcolor{blue}{Expliquer les diff\'erences \'eventuelles de temps calcul (examiner aussi le fichier {\tt scheme.cxx}).}
	\end{itemize}
	 
\end{frame}

\begin{frame}[fragile]
Se mettre dans le r\'epertoire {\tt I03\_TP1}, faire une copie du r\'epertoire {\tt src\_Q3a} dans {Q3a/src} et compiler:
\begin{quote}
	\begin{verbatim}
	mkdir -p Q3a/build
	cp -rf src_Q3a Q3a/src
	cd Q3a/build && cmake ../src && cd ..
	make -C build
	\end{verbatim}
\end{quote}

\vfill
\textcolor{blue}{\bf Q3a. Le fichier {\tt scheme.cxx} a \'et\'e modifi\'e pour pouvoir traiter des \'equations avec ou sans termes de diffusion (laplacien) et d'advection}

\begin{itemize}
	\item \textcolor{blue}{Examiner le fichier {scheme.cxx}: on utilise 2 variables logiques ({\tt bool}) pour incorporer ou non les termes de diffusion-convection dans le calcul}
	\item \textcolor{blue}{Ex\'ecuter le code sous les 4 formes ci-dessus et comparer :}
\end{itemize}
\end{frame}

\begin{frame}[fragile]
		\begin{quote}
		\color{blue}
			\begin{verbatim}
			./build/PoissonSeq
			./build/PoissonSeq convection
			./build/PoissonSeq diffusion
			./build/PoissonSeq convection diffusion
			\end{verbatim}
		\end{quote}
	
\begin{itemize}		
\item		\textcolor{blue}{Comparer et commenter les r\'esultats.}.
\end{itemize}

\end{frame}

\begin{frame}[fragile]
Se mettre dans le r\'epertoire {\tt I03\_TP1}, faire une copie du r\'epertoire {\tt src\_Q3b} dans {Q3b/src} et compiler:
\begin{quote}
	\begin{verbatim}
	mkdir -p Q3b/build
	cp -rf src_Q3b Q3b/src
	cd Q3b/build && cmake ../src && cd ..
	make -C build
	\end{verbatim}
\end{quote}

\vfill
\textcolor{blue}{\bf Q3b. Le fichier {\tt scheme.cxx} contient une variante de l'algorithme vu en Q3a}

\begin{itemize}
	\item \textcolor{blue}{Examiner le fichier {scheme.cxx}: on utilise 2 variables de type ({\tt double}) qui multiplient les termes de convection et diffusion pour les incorporer ou non dans le calcul (ces variables sont \'egales a 0.0 ou a 1.0.}
	\item \textcolor{blue}{Faire les tests analogues au Q3a, comparer et commenter les r\'esultats.}
\end{itemize}
\end{frame}

\begin{frame}[fragile]
\begin{quote}
\color{blue}
\begin{verbatim}
./build/PoissonSeq
./build/PoissonSeq convection
./build/PoissonSeq diffusion
./build/PoissonSeq convection diffusion
\end{verbatim}
\end{quote}

\begin{itemize}		
\item		\textcolor{blue}{Comparer et commenter les r\'esultats.}.
\end{itemize}

\end{frame}

\end{document}