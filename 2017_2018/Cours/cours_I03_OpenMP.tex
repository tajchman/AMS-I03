\documentclass{beamer}
\usepackage[utf8]{inputenc}
\usetheme[]{boxes}
\usecolortheme{seagull}
\usepackage{listings}

\begin{document}
\begin{frame}
	\frametitle{OpenMP :}
	\vfill
	\Large
	Points abord\'es :
	\vfill
	\begin{itemize}
		\item Programmation multi-threads
		\bigskip
		\item Exemples OpenMP
		\bigskip
        \item Styles de programmation OpenMP
	\end{itemize}
	\vfill
\end{frame}

\begin{frame}
\frametitle{Rappels d'architecture}
	
	Exemple d\'ej\`a vu \`a la premi\`ere s\'eance de vue interne d'un n\oe ud (variable suivant le mod\`ele de processeur et la g\'en\'eration utilis\'ee):
	
	\begin{center}
		\includegraphics[scale=0.3]{architecture2}
	\end{center}

\end{frame}
	
\begin{frame}
	\begin{tabular}{lcl}
		Une machine &=& un ou plusieurs processeurs \\
		& + & une m\'emoire centrale ($\approx 10^{10}$ octets) partag\'ee \\
		& & (accessible directement par tous les processeurs)\\ \\
		Un processeur & = & un ou plusieurs c\oe urs \\
		& + & un ensemble de zones m\'emoire rapides ou\\
		& & mémoire cache ($\approx 10^{4}$ \`a $10^{6}$ octets)(caches) \\
		& & partag\'ees entre tous ou une partie des c\oe urs \\
		\\
		Un c\oe ur & = & une unit\'e de traitement capable d'ex\'ecuter \\
		& & un thread (une s\'equence d'instructions)
	\end{tabular}
\end{frame}

\begin{frame}
	\vfill
	En g\'en\'eral, une machine qui contient $N$ processeurs o\`u chaque processeur est compos\'e de $M$ c\oe urs peut ex\'ecuter au maximum $N\times M$ threads (s\'equences d'instructions) simultan\'ement.
	
	\vfill
	La gestion dans un programme de cet ensemble de threads s'appelle la programmation multi-threads
	\vfill
	
	Plusieurs outils logiciels permettent de faire cette gestion:
	\begin{itemize}
		\item pthreads
		\item OpenMP
		\item Threading Building Blocks
		\item ...
	\end{itemize}

	\vfill
	On utilisera ici l'outil OpenMP
\end{frame}

\begin{frame}
	\vfill
	\vfill
\end{frame}
\end{document}
