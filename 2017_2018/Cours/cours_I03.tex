\documentclass{beamer}
\usepackage[utf8]{inputenc}
\usetheme[]{boxes}
\usecolortheme{seagull}
%\usepackage{french}
\title{Modèles et techniques en programmation parallèle hybride et multi-c\oe urs}
\author{Marc Tajchman}\institute{CEA - DEN/DM2S/STMF/LMES}

\begin{document}

\begin{frame}
\titlepage
\end{frame}

\Large
\begin{frame}
  	\frametitle{Plan}
  	\tableofcontents
\end{frame}

\begin{frame}
\section{Optimisation de la programmation séquentielle}
\frametitle{Optimisation de la programmation séquentielle \hbox{(2 séances)}}
Points abordés
\begin{itemize}
\item Modèle d'architecture matérielle
\item Localités spatiale et temporelle (optimisation de l'utilisation de la mémoire cache)
\item Parallélisme à l'intérieur d'un c\oe ur
\item Exemples
\end{itemize}
\end{frame}

\begin{frame}
\frametitle{Modèle d'architecture matérielle}
\end{frame}

\begin{frame}
\frametitle{Localité spatiale}
Règle: autant que possible, utiliser des zones mémoires proches les unes des autres dans une séquence d'instructions

\bigskip
But: réduire la fréquence de transferts mémoire centrale - mémoire cache
\end{frame}

\begin{frame}
\frametitle{Localité temporelle}
Règle: autant que possible, pour une zone mémoire, les instructions qui l'utilisent doivent s'exécuter de façon rapprochée

\bigskip
But: réduire la fréquence de transferts mémoire centrale - mémoire cache
\end{frame}

\begin{frame}
\section{Rappels de programmation parallèle}
\subsection{Rappel des notions}
\frametitle{Rappels de programmation parallèle: notions}
Points abordés
\begin{itemize}
\item mémoire distribuée
\item mémoire partagée
\item threads
\item processus
\end{itemize}
\end{frame}

\begin{frame}
\subsection{Mémoire partagée}
\frametitle{Rappels de programmation parallèle: mémoire partagée}
Points abordés
\begin{itemize}
\item Modèle d'architecture matérielle
\item Principes d'optimisation
\item Cas classique : OpenMP, pthreads
\item Autres
\end{itemize}
\end{frame}

\begin{frame}
\subsection{Mémoire distribuée}
\frametitle{Rappels de programmation parallèle: mémoire distribuée}
Points abordés
\begin{itemize}
\item Modèle d'architecture matérielle
\item Principes d'optimisation
\item Cas classique : MPI
\item Autres
\end{itemize}
\end{frame}

\begin{frame}
\section{Programmation parallèle hybride}
\frametitle{Programmation parallèle hybride \hbox{(4 séances)}}
Points abordés
\begin{itemize}
\item Coexistence 
\item Modèles d'hybridation
\item Cas classique : MPI - OpenMP
\item Exemples
\item Autres modèles (e.g. MPI+X, PGAS)
\end{itemize}
\end{frame}

\begin{frame}
\section{Examen}
\frametitle{Examen \hbox{(1 séance)}}
\end{frame}

\end{document}
