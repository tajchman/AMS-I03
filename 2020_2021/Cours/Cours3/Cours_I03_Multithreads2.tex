\documentclass{beamer}
\usepackage[utf8]{inputenc}
\usepackage{graphicx}
\usepackage{listings}
\usepackage{hyperref}
\hypersetup{
	colorlinks=true,
	linkcolor=blue,
	filecolor=magenta,      
	urlcolor=cyan,
}

\urlstyle{same}

\usepackage{xcolor}

\lstdefinestyle{base}{
	language=C++,
	emptylines=1,
	breaklines=true,
	basicstyle=\ttfamily\color{black},
	moredelim=**[is][\bf\color{red}]{@}{@},
}

\usetheme[]{boxes}
\usecolortheme{seagull}
\addtobeamertemplate{navigation symbols}{}{%
	\usebeamerfont{footline}%
	\usebeamercolor[fg]{footline}%
	\hspace{2em}%
	\insertframenumber/\inserttotalframenumber
}

%\usepackage{french}
\title{Modèles et techniques en programmation parallèle hybride et multi-c\oe urs}
\subtitle{Parall\'elisme multithreads (2)}
\author{Marc Tajchman}\institute{CEA - DEN/DM2S/STMF/LMES}
\date{10/08/2020}

\begin{document}
\begin{frame}
	\titlepage
\end{frame}

\large
\begin{frame}
	\section{Techniques de programmation avec OpenMP}
	\frametitle{Techniques de programmation avec OpenMP}

\begin{itemize}
	\item Programmation OpenMP grain fin
	\bigskip
	\item Programmation OpenMP grain grossier
	\bigskip
	\item Programmation OpenMP par tâches
\end{itemize}
\end{frame}

\begin{frame}
	La programmation OpenMP vue dans les exemples jusqu'à present est de type ``grain fin'' : c'est OpenMP qui répartit le travail dans les régions parallèles entre les threads.
	
	Le programmeur a assez peu de possibilités pour adapter la programmation à un problème particulier
\end{frame}

\end{document}
