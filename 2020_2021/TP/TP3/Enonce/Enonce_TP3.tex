\documentclass{beamer}
\usepackage[utf8]{inputenc}
\usepackage{graphicx}
\usepackage{listings}
\usepackage{hyperref}
\hypersetup{
	colorlinks=true,
	linkcolor=blue,
	filecolor=magenta,      
	urlcolor=cyan,
}

\urlstyle{same}

\usepackage{xcolor}

\lstdefinestyle{base}{
	language=C++,
	emptylines=1,
	breaklines=true,
	basicstyle=\ttfamily\color{black},
	moredelim=**[is][\bf\color{red}]{@}{@},
}

\usetheme[]{boxes}
\usecolortheme{seagull}
\addtobeamertemplate{navigation symbols}{}{%
	\usebeamerfont{footline}%
	\usebeamercolor[fg]{footline}%
	\hspace{2em}%
	\insertframenumber/\inserttotalframenumber
}

\newcommand\Frac[2]{\frac{\displaystyle #1}{\displaystyle #2}}

%\usepackage{french}
\title{Modèles et techniques en programmation parallèle hybride et multi-c\oe urs}
\subtitle{Travail pratique 3}
\author{Marc Tajchman}\institute{CEA - DEN/DM2S/STMF/LMES}
\date{mise à jour le 10/01/2021}

\begin{document}
\begin{frame}
	\titlepage
\end{frame}

\large
\begin{frame}
	\section{Travail pratique 2}
	\frametitle{Travail pratique 2}

On part de deux code qui calculent une solution approchée du problème suivant~:

\medskip
\begin{quote}
Chercher $u$:  $(x, t) \mapsto u(x, t)$, où  $x \in \Omega = [0,1]^3$ et $t \geq 0$, qui vérifie :
$$
\begin{array}{lcll}
\Frac{\partial u}{\partial t} & = & \Delta u + f(x, t) & \\[0.3cm]
u(x, 0) &=& g(x) & x\in \Omega \\[0.3cm]
u(x, t) & = & g(x) & x\in\partial \Omega, t > 0\\[0.3cm]
\end{array}
$$

\vspace{-0.6cm}
où $f$ et $g$ sont des fonctions données.
\end{quote}

Le code utilise des différences finies pour approcher les dérivées partielles et découpe $\Omega$ en $n_0\times n_1\times n_2$ subdivisions.

\end{frame}

\begin{frame}
	\frametitle{Codes de départ}
	
   \vfill
	Le premier code utilise la parallélisation avec MPI et la décomposition de domaines
    (ce code parallélisée est similaire au code MPI utilisé dans le TP2)
	
	\medskip
	
	Récupérer et décompresser un des fichiers \href{https://perso.ensta-paris.fr/~tajchman/Seance9/TP3_MPI.tar.gz}{\tt TP3\_MPI.tar.gz} ou \href{https://perso.ensta-paris.fr/~tajchman/Seance9/TP3_MPI.zip}{\tt TP3\_MPI.zip}.
	
   \vfill
    
    Le second code utilise la parallélisation avec Cuda sur carte graphique.
    
	\medskip
	
	Récupérer et décompresser un des fichiers \href{https://perso.ensta-paris.fr/~tajchman/Seance9/TP3_Cuda.tar.gz}{\tt TP3\_Cuda.tar.gz} ou \href{https://perso.ensta-paris.fr/~tajchman/Seance9/TP3_Cuda.zip}{\tt TP3\_Cuda.zip}.

   \vfill
\end{frame}

\begin{frame}
\medskip
  
\textbf{Le but du TP est d'obtenir une version hybride destinée à être utilisée sur une machine parallèle dont chaque n\oe ud contient une carte graphique.}
  
\medskip

  On pourra utiliser la machine rhum.ensta.fr, qui n'a qu'un seul n\oe ud mais qui contient 2 cartes graphiques.
  
  On comparera les performances du code :
  \begin{itemize}
  	\item avec 1 processus MPI, en utilisant ou pas une carte graphique 
  	\item avec 2 processus MPI, chacun utilisant une des 2 cartes graphiques 
  \end{itemize}
  
   
\vfill
\end{frame}


\begin{frame}

\bigskip
Envoyez par mail à \href{mailto:marc.tajchman@cea.fr}{marc.tajchman@cea.fr} :

\begin{itemize}
	\item une description du travail réalisé (1-2 pages maximum)
	\item le code source, avec vos modifications, dans une archive
	(n'envoyez pas les répertoires \textcolor{blue}{\tt build} et  \textcolor{blue}{\tt install} qui contiennent des binaires) 
	\item autant que possible, les fichiers qui contiennent les sorties écran.
\end{itemize}

\bigskip

{\bf avant le 14/02/2021.}

\bigskip
\textcolor{red}{\bf Envoyez vos fichiers source même s'ils contiennent des erreurs.}
\end{frame}


\end{document}
