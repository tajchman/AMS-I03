\documentclass{beamer}
\usepackage[utf8]{inputenc}
\usepackage{graphicx}
\usepackage{listings}
\usepackage{hyperref}
\hypersetup{
	colorlinks=true,
	linkcolor=blue,
	filecolor=magenta,      
	urlcolor=cyan,
}

\urlstyle{same}

\usepackage{xcolor}

\lstdefinestyle{base}{
	language=C++,
	emptylines=1,
	breaklines=true,
	basicstyle=\ttfamily\color{black},
	moredelim=**[is][\bf\color{red}]{@}{@},
}

\usetheme[]{boxes}
\usecolortheme{seagull}
\addtobeamertemplate{navigation symbols}{}{%
	\usebeamerfont{footline}%
	\usebeamercolor[fg]{footline}%
	\hspace{2em}%
	\insertframenumber/\inserttotalframenumber
}

\newcommand\Frac[2]{\frac{\displaystyle #1}{\displaystyle #2}}

%\usepackage{french}
\title{Modèles et techniques en programmation parallèle hybride et multi-c\oe urs}
\subtitle{Travail pratique 3}
\author{Marc Tajchman}\institute{CEA - DEN/DM2S/STMF/LMES}
\date{01/02/2021}

\begin{document}
\begin{frame}
	\titlepage
\end{frame}

\large
\begin{frame}
	\section{Travail pratique 3}
	\frametitle{Travail pratique 3}

On part de deux codes qui calculent une solution approchée du problème suivant~:

\medskip
\begin{quote}
Chercher $u$:  $(x, t) \mapsto u(x, t)$, où  $x \in \Omega = [0,1]^3$ et $t \geq 0$, qui vérifie :
$$
\begin{array}{lcll}
\Frac{\partial u}{\partial t} & = & \Delta u + f(x, t) & \\[0.3cm]
u(x, 0) &=& g(x) & x\in \Omega \\[0.3cm]
u(x, t) & = & g(x) & x\in\partial \Omega, t > 0\\[0.3cm]
\end{array}
$$

\vspace{-0.6cm}
où $f$ et $g$ sont des fonctions données.
\end{quote}

On utilise des différences finies pour approcher les dérivées partielles et on découpe $\Omega$ en $n_0\times n_1\times n_2$ subdivisions.

\end{frame}

\begin{frame}
	\frametitle{Codes de départ}
		
	Récupérer et décompresser le fichier \href{https://perso.ensta-paris.fr/~tajchman/Seance9/TP3.tar.gz}{\tt TP3.tar.gz}.

	\medskip

    Cette archive contient 4 sous-répertoires : 
    \begin{itemize}
    	\item la version séquentielle (\texttt{PoissonSeq}),
    	\item la version parallélisée avec OpenMP (grain fin) (\texttt{PoissonOpenMP}),
    	\item la version parallélisée avec MPI (\texttt{PoissonMPI}),
    	\item la version parallélisée avec Cuda (\texttt{PoissonCuda}),
    	
    \end{itemize}

	\medskip
   Les deux premières versions servent de référence.
   
   La version MPI est similaire à celle utilisée dans le TP 2.
   \vfill
\end{frame}

\begin{frame}
\medskip
 
Le but du TP est d'étudier la fusion entre les deux versions Cuda et MPI, pour obtenir une version hybride destinée à être utilisée sur une machine parallèle dont chaque n\oe ud contient une carte graphique.
  
 \vfill
On propose de faire une copie du code MPI et d'y ajouter les fonctions Cuda (en les adaptant) de la version Cuda.

\vfill
On demande un rapport de 2-3 pages pour décrire les modifications du code MPI pour faire les calculs internes dans chaque sous-domaine MPI par une carte graphique.

   \vfill

\end{frame}

\begin{frame}

Décrivez ce que vous proposez de modifier:
\begin{itemize}
	\item fichiers et fonctions concernés
	\item impact sur les structures de données
	\item où sont faits les calculs : processeur CPU ou GPU
	\item quelles sont les copies de données entre les CPU, entre un CPU et le GPU associé,
	\item etc.
\end{itemize}

\vfill

On ne demande pas d'écrire effectivement le code de la version hybride (mais ceux qui veulent essayer sont les bienvenus et je pourrai les aider).
   
   \vfill
Envoyez votre rapport par mail à \href{mailto:marc.tajchman@cea.fr}{marc.tajchman@cea.fr}
{\bf avant le 14/02/2021.}

\end{frame}


\end{document}
