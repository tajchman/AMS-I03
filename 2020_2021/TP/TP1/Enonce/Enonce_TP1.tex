\documentclass{beamer}
\usepackage[utf8]{inputenc}
\usepackage{graphicx}
\usepackage{listings}
\usepackage{hyperref}
\hypersetup{
	colorlinks=true,
	linkcolor=blue,
	filecolor=magenta,      
	urlcolor=cyan,
}

\urlstyle{same}

\usepackage{xcolor}

\lstdefinestyle{base}{
	language=C++,
	emptylines=1,
	breaklines=true,
	basicstyle=\ttfamily\color{black},
	moredelim=**[is][\bf\color{red}]{@}{@},
}

\usetheme[]{boxes}
\usecolortheme{seagull}
\addtobeamertemplate{navigation symbols}{}{%
	\usebeamerfont{footline}%
	\usebeamercolor[fg]{footline}%
	\hspace{2em}%
	\insertframenumber/\inserttotalframenumber
}

\newcommand\Frac[2]{\frac{\displaystyle #1}{\displaystyle #2}}

%\usepackage{french}
\title{Modèles et techniques en programmation parallèle hybride et multi-c\oe urs}
\subtitle{Travail pratique n°1}
\author{Marc Tajchman}\institute{CEA - DEN/DM2S/STMF/LMES}
\date{02/12/2020}

\begin{document}
\begin{frame}
	\titlepage
\end{frame}

\large
\begin{frame}
	\section{Travail pratique n°1}
	\frametitle{Travail pratique n°1}

A partir d'un code séquentiel qui calcule une solution approchée du problème suivant~:

\medskip
\begin{quote}
Chercher $u$:  $(x, t) \mapsto u(x, t)$, où  $x \in \Omega = [0,1]^3$ et $t \geq 0$, qui vérifie :
$$
\begin{array}{lcll}
\Frac{\partial u}{\partial t} & = & \Delta u + f(x, t) & \\[0.3cm]
u(x, 0) &=& g(x) & x\in \Omega \\[0.3cm]
u(x, t) & = & g(x) & x\in\partial \Omega, t > 0\\[0.3cm]
\end{array}
$$

\vspace{-0.6cm}
où $f$ et $g$ sont des fonctions données.
\end{quote}

Le code séquentiel utilise des différences finies pour approcher les dérivées partielles et découpe $\Omega$ en $n_1\times n_2\times n_3$ subdivisions.

Récupérer le fichier \href{https://perso.ensta-paris.fr/~tajchman/Seance3/TP1_sequentiel.tar.gz}{\tt TP1\_sequentiel.tar.gz}
\end{frame}

\begin{frame}
	\frametitle{Structure du code séquentiel}
	
	Récupérer et décompresser un des fichiers \href{https://perso.ensta-paris.fr/~tajchman/Seance3/TP1_sequentiel.tar.gz}{\tt TP1\_sequentiel.tar.gz} ou \href{https://perso.ensta-paris.fr/~tajchman/Seance3/TP1_sequentiel.zip}{\tt TP1\_sequentiel.zip}.
	
    \medskip
	 Se placer dans le répertoire {\tt TP1/PoissonSeq} créé. 
	
	Le code séquentiel est réparti en plusieurs fichiers principaux dans le sous-répertoire {\tt src}:
    \medskip
    
	\hfill\begin{minipage}{8cm}
	\begin{itemize}
		\item[\textcolor{blue}{main.cxx}:] programme principal: initialise, appelle le calcul des itérations en temps, affiche les résultats 
		\item[\textcolor{blue}{scheme(.hxx/.cxx)}:] définit le type {\tt Scheme} qui calcule une itération en temps
		\item[\textcolor{blue}{values(.hxx/.cxx)}:] définit le type {\tt Values} qui contient les valeurs approchées à un instant donné
		\item[\textcolor{blue}{parameters(.hxx/.cxx)}:] définit le type {\tt Parameters} qui rassemble les informations sur la géométrie et le calcul
	\end{itemize}
    \end{minipage}
\vfill
\end{frame}

\begin{frame}
Fonctions du type {\tt Scheme} :
\bigskip

\begin{tabular}{ll}
{\tt Scheme(P, f)} & construit une variable de type {\tt Scheme} \\
& en lui donnant une variable de type {\tt Parameters} \\
& et une fonction {\tt f} (second membre de l'équation) \\ \\
{\tt iteration()} & calcule une itération (la valeur de la \\ 
& solution à l'instant suivant)\\
{\tt variation()} & retourne la variation entre 2 instants \\ 
& de calcul successifs\\ \\

{\tt getOutput()} & renvoie une variable de type {\tt Values} qui \\ & contient les dernières valeurs calculées \\
{\tt setInput(u)} & rentre dans {\tt Scheme} les valeurs initiales \\
\end{tabular}
	
\end{frame}

\begin{frame}
	Fonctions du type {\tt Parameters} :
	\bigskip
	
	\begin{tabular}{ll}
		{\tt n(i)} & nombre de points dans la direction $i$ \\
		&  ($0=X$,$1=Y$, $2=Z$), y compris sur la frontière \\
		{\tt imin(i)}& indice des premiers points intérieurs \\ &  dans la direction $i$\\
		{\tt imax(i)}& indice des derniers points sur la frontière \\ & dans la direction $i$ \\ \\
		{\tt dx(i)} & dimension d'une subdivision dans la direction $i$\\
		{\tt xmin(i)} & coordonnée minimale de $\Omega$ dans la direction $i$ \\ \\
		{\tt itmax()} & nombre d'itérations en temps \\
		{\tt dt()} & intervalle de temps entre 2 itérations \\
		{\tt freq()} & fréquence de sortie des résultats intermédiaires \\
		& (nombre d'itérations entre 2 sorties)
	\end{tabular}
	
\end{frame}

\begin{frame}
	
\vfill
Les points de calcul à l'intérieur du domaine $\Omega$ ont des indices $(i,j,k)$ tels que:
\bigskip

\begin{tabular}{lclcl}
	{\tt imin(0)} & $\leq$ & {\tt i} & $<$ & {\tt imax(0)} \\
	{\tt imin(1)} & $\leq$ & {\tt j} & $<$ & {\tt imax(1)} \\
	{\tt imin(2)} & $\leq$ & {\tt k} & $<$ & {\tt imax(2)} \\
\end{tabular}
\vfill

Les points sur la frontière du domaine $\partial\Omega$ ont des indices $(i,j,k)$ tels que:
\bigskip

\begin{tabular}{lcl}
	{\tt i = imin(0)-1} & ou & {\tt i = imax(0)}  \\
	{\tt j = imin(j)-1} & ou & {\tt j = imax(1)}  \\
	{\tt k = imin(k)-1} & ou & {\tt k = imax(2)} 
\end{tabular}
\vfill
	
	
\end{frame}

\begin{frame}
Fonctions du type {\tt Values}:
\bigskip

\begin{tabular}{ll}
	{\tt init()} & initialise les points du domaine à 0 \\
	{\tt init(f)}& initialise les points du domaine \\ & avec la fonction $f : (x,y,z) \mapsto f(x,y,z)$\\
	{\tt boundaries(g)} & initialise les points de la frontière \\ &  avec la fonction $g: (x,y,z) \mapsto g(x,y,z)$\\ \\
	{\tt v(i,j,k)}& si {\tt v} est de type {Values}, la valeur au point \\ &  d'indice $(i,j,k)$ \\
	{\tt v.swap(w)} & si {\tt v} et {\tt w} sont de type {\tt Values}, \\ & échange les valeurs de {\tt v} et {\tt w} \\ \\
\end{tabular}

\end{frame}

\begin{frame}[fragile]
\vfill

	
\begin{itemize}
\item 	Pour compiler, se placer dans le répertoire {\tt PoissonSeq} et taper:

\hspace{2cm}{\color{blue}\begin{verbatim}
./build.py
\end{verbatim}
}
		
\vfill
		(si cela ne marche pas, taper \verb|python ./build.py|).
		
\vfill
	\item Pour exécuter, rester dans le même répertoire et taper:

\hspace{2cm}
{\color{blue}\begin{verbatim}
./install/gnu/Release/PoissonSeq.exe
\end{verbatim}
}

\vfill
Pour voir les options d'exécution possibles, taper \verb|./install/gnu/Release/PoissonSeq.exe --help|
\end{itemize}

\vfill
Noter les valeurs obtenues et les temps de calcul affichés, ils serviront de référence pour évaluer les autres versions. 
\vfill
\end{frame}

\begin{frame}[fragile]
	\frametitle{Version multithreads avec OpenMP (grain fin)}
	
Récupérer et décompresser un des fichiers \bigskip

\href{https://perso.ensta-paris.fr/~tajchman/Seance3/TP1_OpenMP_FineGrain_incomplet.tar.gz}{\tt TP1\_OpenMP\_FineGrain\_incomplet.tar.gz} ou \href{https://perso.ensta-paris.fr/~tajchman/Seance3/TP1_OpenMP_FineGrain_incomplet.zip}{\tt TP1\_OpenMP\_FineGrain\_incomplet.zip}
\bigskip

Un répertoire {\tt TP1/Poisson\_OpenMP\_FineGrain} est créé et contient le code source incomplet de la version OpenMP grain fin.
 
\vfill
{\color{red}
	Attention:
	\bigskip
	
	\begin{minipage}{\textwidth}\color{red}
Si vous avez déjà récupéré ce fichier et modifié les sources qui y sont contenues, créez un autre répertoire ailleurs et travaillez dans ce nouveau répertoire, sinon vous écraserez les modifications déjà faites !
	\end{minipage}
}
\vfill
\end{frame}

\begin{frame}[fragile]

\begin{itemize}
	\item 	Pour compiler, se placer dans le répertoire {\tt TP1/Poisson\_OpenMP\_FineGrain} et taper:
	
	\hspace{-2cm}{\color{blue}\begin{verbatim}
./build.py
\end{verbatim}
	}
	
	\vfill
	(si cela ne marche pas, taper \verb|python ./build.py|).
	
	\vfill
	\item Pour exécuter, rester dans le même répertoire et taper (sur une seule ligne):
	
	\hspace{2cm}
	{\color{blue}\begin{verbatim}
./install/gnu/Release/PoissonOpenMP_FineGrain.exe \
		 threads=<n>
\end{verbatim}
	}
	
	\vfill
	(à la place de {\tt <n>} taper 3 pour exécuter sur 3 threads (par exemple)
	\vfill
	
\end{itemize}
\end{frame}


\begin{frame}[fragile]
	On fournit aussi un script pour comparer les speedups pour différents nombres de threads:
	
	{\color{blue}\begin{verbatim}
		./run.py <nthreads>
		\end{verbatim}
	}
	qui lance plusieurs exécutions:
	\begin{itemize}
		\item version séquentielle
		\item version OpenMP sur 1 thread
		\item version OpenMP sur 2 threads
		
		...
		\item version OpenMP sur \verb|<nthreads>| threads
	\end{itemize}
	
	et à la fin affiche un graphe: nombre de threads .vs. temps CPU/speedup.
	
	\vfill
	Remarque:
	\medskip
	
	\hspace{1cm}\begin{minipage}{0.9\textwidth}
		Pour pouvoir utiliser le script {\tt run.py},
		il faut que le paquet {\tt matplotlib} (pour la version de python utilisée) soit installé
	\end{minipage}
	
\end{frame}

\begin{frame}
	Première partie:
\bigskip

\framebox{\color{blue}%
\begin{minipage}{\textwidth}
		Dans le répertoire {\tt TP1/Poisson\_OpenMP\_FineGrain}, paral\-léliser avec OpenMP grain fin:
		
		\begin{enumerate}
			\item \color{blue}Chercher les parties du code à paralléliser
			\item \color{blue}Ajouter ou adapter les {\tt pragmas}
			\item \color{blue}Identifier les variables partagées et privées
			\item \color{blue}Compiler, lancer le code avec différents nombres de threads
			\item \color{blue}Comparer avec la version séquentielle
			\item \color{blue}Si différent, revenir en (2)
		\end{enumerate}
	
	\end{minipage}
}

\end{frame}

\begin{frame}[fragile]
	\frametitle{Version multithreads avec OpenMP (grain grossier)}
	
	Récupérer et décompresser un des fichiers \bigskip
	
	\href{https://perso.ensta-paris.fr/~tajchman/Seance3/TP1_OpenMP_FineGrain_incomplet.tar.gz}{\tt TP1\_OpenMP\_CoarseGrain\_incomplet.tar.gz} ou \href{https://perso.ensta-paris.fr/~tajchman/Seance3/TP1_OpenMP_FineGrain_incomplet.zip}{\tt TP1\_OpenMP\_CoarseGrain\_incomplet.zip}
	\bigskip
	
	Un répertoire {\tt TP1/Poisson\_OpenMP\_CoarseGrain} est créé et contient le code source incomplet de la version OpenMP grain fin.
	
	\vfill
	Pour compiler et exécuter on utilisera la même procédure que dans le cas OpenMP grain fin.
	
\end{frame}


\begin{frame}
	Dans le parallélisme OpenMP gros grain, on découpe explicitement le domaine de calcul en plusieurs parties, chaque partie est calculée par un thread.
	\vfill
	
	Le type {\tt Parameters} possède 2 fonctions supplémentaires utiles pour le // gros grain :
	\bigskip
	
	\begin{tabular}{ll}
		{\tt imin\_local(i,iThread)}&indice des premiers points intérieurs \\ &  dans la direction $i$, pour la partie \\ 
		&calculée par le thread {\tt iThread}\\ \\
		{\tt imax\_local(i,iThread)}& indice des derniers points sur la \\
		& frontière dans la direction $i$, pour la \\
		& partie calculée par le thread {\tt iThread}
	\end{tabular}
	
\end{frame}

\begin{frame}
	Seconde partie:
	\bigskip
	
	\framebox{\color{blue}%
		\begin{minipage}{\textwidth}
			Dans le répertoire {\tt TP1/Poisson\_OpenMP\_CoarseGrain}, paralléliser avec OpenMP grain grossier:
			
			\begin{enumerate}
				\item \color{blue}Ajouter une région parallèle dans le programme principal autour de la boucle en temps
				\item \color{blue}Identifier les instructions dans la région parallèle qui doivent s'exécuter en séquentiel et celles qui peuvent s'exécuter en parallèle
				\item \color{blue}Identifier les variables partagées et privées
				\item \color{blue}Ajouter ou adapter les pragmas correspondantes
				\item \color{blue}Dans chaque thread, faire une partie des boucles internes
				\item \color{blue}Compiler, lancer le code avec différents nombres de threads
				\item \color{blue}Comparer avec la version séquentielle
				\item \color{blue}Si différent, revenir en (2)
			\end{enumerate}
		\end{minipage}
	}
	
\end{frame}

\begin{frame}
\bigskip
Envoyez par mail à \href{mailto:marc.tajchman@cea.fr}{marc.tajchman@cea.fr} :

\begin{itemize}
	\item une description du travail réalisé (1-2 pages maximum)
	\item le code source, avec vos modifications, dans une archive
	(n'envoyez pas les répertoires \textcolor{blue}{\tt build} et  \textcolor{blue}{\tt install} qui contiennent des binaires) 
	\item les fichiers run***.log et speedups***.pdf que vous avez obtenus
\end{itemize}

\bigskip

{\bf avant le 20/12/2020, au plus tard le 23/12/2020.}

\bigskip
\textcolor{red}{\bf Envoyez vos fichiers source même s'ils contiennent des erreurs.}
\end{frame}


\end{document}
